\documentclass[]{article}
\usepackage{lmodern}
\usepackage{amssymb,amsmath}
\usepackage{ifxetex,ifluatex}
\usepackage{fixltx2e} % provides \textsubscript
\ifnum 0\ifxetex 1\fi\ifluatex 1\fi=0 % if pdftex
  \usepackage[T1]{fontenc}
  \usepackage[utf8]{inputenc}
\else % if luatex or xelatex
  \ifxetex
    \usepackage{mathspec}
  \else
    \usepackage{fontspec}
  \fi
  \defaultfontfeatures{Ligatures=TeX,Scale=MatchLowercase}
\fi
% use upquote if available, for straight quotes in verbatim environments
\IfFileExists{upquote.sty}{\usepackage{upquote}}{}
% use microtype if available
\IfFileExists{microtype.sty}{%
\usepackage{microtype}
\UseMicrotypeSet[protrusion]{basicmath} % disable protrusion for tt fonts
}{}
\usepackage[margin=1in]{geometry}
\usepackage{hyperref}
\hypersetup{unicode=true,
            pdftitle={DISCOVER ASSOCIATIONS BETWEEN PRODUCTS},
            pdfauthor={Kevin Bergmeijer and Alejandro Rojo},
            pdfborder={0 0 0},
            breaklinks=true}
\urlstyle{same}  % don't use monospace font for urls
\usepackage{graphicx}
% grffile has become a legacy package: https://ctan.org/pkg/grffile
\IfFileExists{grffile.sty}{%
\usepackage{grffile}
}{}
\makeatletter
\def\maxwidth{\ifdim\Gin@nat@width>\linewidth\linewidth\else\Gin@nat@width\fi}
\def\maxheight{\ifdim\Gin@nat@height>\textheight\textheight\else\Gin@nat@height\fi}
\makeatother
% Scale images if necessary, so that they will not overflow the page
% margins by default, and it is still possible to overwrite the defaults
% using explicit options in \includegraphics[width, height, ...]{}
\setkeys{Gin}{width=\maxwidth,height=\maxheight,keepaspectratio}
\IfFileExists{parskip.sty}{%
\usepackage{parskip}
}{% else
\setlength{\parindent}{0pt}
\setlength{\parskip}{6pt plus 2pt minus 1pt}
}
\setlength{\emergencystretch}{3em}  % prevent overfull lines
\providecommand{\tightlist}{%
  \setlength{\itemsep}{0pt}\setlength{\parskip}{0pt}}
\setcounter{secnumdepth}{0}
% Redefines (sub)paragraphs to behave more like sections
\ifx\paragraph\undefined\else
\let\oldparagraph\paragraph
\renewcommand{\paragraph}[1]{\oldparagraph{#1}\mbox{}}
\fi
\ifx\subparagraph\undefined\else
\let\oldsubparagraph\subparagraph
\renewcommand{\subparagraph}[1]{\oldsubparagraph{#1}\mbox{}}
\fi

%%% Use protect on footnotes to avoid problems with footnotes in titles
\let\rmarkdownfootnote\footnote%
\def\footnote{\protect\rmarkdownfootnote}

%%% Change title format to be more compact
\usepackage{titling}

% Create subtitle command for use in maketitle
\providecommand{\subtitle}[1]{
  \posttitle{
    \begin{center}\large#1\end{center}
    }
}

\setlength{\droptitle}{-2em}

  \title{DISCOVER ASSOCIATIONS BETWEEN PRODUCTS}
    \pretitle{\vspace{\droptitle}\centering\huge}
  \posttitle{\par}
    \author{Kevin Bergmeijer and Alejandro Rojo}
    \preauthor{\centering\large\emph}
  \postauthor{\par}
      \predate{\centering\large\emph}
  \postdate{\par}
    \date{11/12/2019}


\begin{document}
\maketitle

\hypertarget{introduction}{%
\section{Introduction}\label{introduction}}

This report will show the \emph{basket market analysis} to better
understand the clientele that Electronidex -- a start-up electronics
online retailer- is currently serving. The final goal is advice and
discuss if Electronidex would be an optimal acquisition based on the
findings shown in this report.

\hypertarget{objectives}{%
\paragraph{Objectives}\label{objectives}}

The objectives of this inform are the following:

\begin{itemize}
\tightlist
\item
  Find patterns or item relationships within transactions.
\item
  Debate if should Blackwell acquire Electronidex.
\item
  Make recommendations about this possible acquisition.
\end{itemize}

\hypertarget{materials}{%
\paragraph{Materials}\label{materials}}

For the development of this exercise the data sets provided by Danielle
Sherman via e-mail in .csv and pdf. formats will be used, consisting of
the following documents:

\begin{itemize}
\tightlist
\item
  \emph{ElectronidexTransactions2017.csv} -- This csv file contains a
  record of one month (30 days) of 9835 online transactions and which
  items were purchased out of the 125 products that Electronidex sells.
\item
  \emph{ElectronidexItems2017.pdf} -- This pdf file contains a list of
  the 125 products that Electronidex sells broken down into 17 product
  types.
\item
  \emph{existingprodutattributes2017.csv} -- This csv file contains
  information about product features, reviews and historical sales
  information. This file has been rescued from the last task.
\end{itemize}

\hypertarget{data-exploration}{%
\section{Data Exploration}\label{data-exploration}}

For the development of this analysis, initially the databases have been
imported and worked on a sparse matrix that included 98355 observations
(transactions) on 125 attributes (product types). It supposes 43104
products bought in one month by Electronidex.

\hypertarget{cleaning-data}{%
\paragraph{Cleaning data}\label{cleaning-data}}

It has been removed to transactions because their respective rows
(\emph{8707}\&\emph{9506}) were empty (missing data).

\hypertarget{categorizing-the-dataset}{%
\paragraph{Categorizing the dataset}\label{categorizing-the-dataset}}

In a first approximation, the product names have been changed by their
product types, keeping the 17 original categories given by Electronidex.
Also, the items have been listed by brand to explore the data. It was
tried to make a \emph{gaming} category, but in a later analysis was
noticed that there were no relevant results.

\includegraphics{Discover_associations_between_products_files/figure-latex/unnamed-chunk-2-1.pdf}
\includegraphics{Discover_associations_between_products_files/figure-latex/unnamed-chunk-2-2.pdf}

\hypertarget{creating-new-groups-retailers-and-corporate}{%
\section{Creating new groups: retailers and
corporate}\label{creating-new-groups-retailers-and-corporate}}

Before starting the search of rules, has been observed that clients are
divided in two classes by the kind of buy that they make: retailers and
corporate. For example, some customers bought 7 desktops or 3 printers.
That is not a normal way to electronic products for a person, but it can
be understood in a company context. So, for a better analysis, the
sample has been split in that groups. For do it, four product types were
categorized as \textbf{\emph{Main items}} (\emph{Laptops, Desktops,
Tablet} and \emph{Printer}). The rest of the products were categorized
as \textbf{\emph{Extra items}}.

The following rules were established to catalogue transactions:

\begin{itemize}
\tightlist
\item
  If the transaction included more than one main item, it was listed as
  corporate (B2B).
\item
  If the transaction included four extra items, it was also listed as
  corporate (B2B).
\item
  All other transactions were catalogued as retailers (B2C).
\end{itemize}

Has been found that 5327 of the transactions were B2B (54,17\%) and 4506
were B2C (45,83\%). Talking about the number of total products sold, B2B
suppose 34867 (80,89\%) of the sold items and B2C the other 8237
(19,11\%).

\includegraphics{Discover_associations_between_products_files/figure-latex/unnamed-chunk-4-1.pdf}
\includegraphics{Discover_associations_between_products_files/figure-latex/unnamed-chunk-4-2.pdf}
\includegraphics{Discover_associations_between_products_files/figure-latex/unnamed-chunk-4-3.pdf}
\includegraphics{Discover_associations_between_products_files/figure-latex/unnamed-chunk-4-4.pdf}

\hypertarget{retail-transactions}{%
\section{Retail transactions}\label{retail-transactions}}

To create the rules, following parameters has been set:

\begin{itemize}
\tightlist
\item
  Support=0.02
\item
  Confidence=0.02
\end{itemize}

The redundant rules were removed from the results. Here are shown the
most relevant rules sort by lift, support and confidence. Any relevant
rule about market basket associations has been noticed for the retail
transactions.

\includegraphics{Discover_associations_between_products_files/figure-latex/unnamed-chunk-5-1.pdf}
\includegraphics{Discover_associations_between_products_files/figure-latex/unnamed-chunk-5-2.pdf}

\hypertarget{corporate-transacions}{%
\section{Corporate transacions}\label{corporate-transacions}}

To create the rules, following parameters has been set:

\begin{itemize}
\tightlist
\item
  Support=0.02
\item
  Confidence=0.02
\end{itemize}

The redundant rules were removed from the results. Here are shown the
most relevant rules sort by lift, support and confidence.

\includegraphics{Discover_associations_between_products_files/figure-latex/unnamed-chunk-6-1.pdf}
\includegraphics{Discover_associations_between_products_files/figure-latex/unnamed-chunk-6-2.pdf}

\textbf{Any relevant rule about market basket associations has been
noticed for the corporate transactions.}

\hypertarget{conclusions}{%
\section{Conclusions}\label{conclusions}}

\begin{itemize}
\tightlist
\item
  First of all, this dataset used is a sample of the transactions made
  in one month by Electronidex. The results are not considered as
  absolute because they can be biased. Therefore, these conclusions
  should be thought of as something indicative.
\item
  Have been observed some rules as a result of the market basket
  associations. These were considered Irrelevant Rules because they were
  not helpful or were obvious. If the dataset will be expanded with a
  greater number of transactions and more qualitative / quantitative
  information about the transactions (price or location of the client,
  for example) it should be re-analysed to find relevant results.
\item
  Electronidex is a company in which more than half of sales are
  corporate (B2B). Blackwell does not focus on this kind of customer
  (store sales weigh heavily, as seen in previous reports). It is
  considered interesting this way as a possibility to expand the
  business and diversify the type of customers. On the other hand, B2B
  sales have lower profit margins.
\item
  It has been observed that in the Electronidex product portfolio there
  are a large number of items that are catalogued as for gamers. It is
  considered to be a booming market (e-Sports outperform other classic
  sports in viewers). The third best-selling product by Blackwell is a
  console, so it makes sense to sell other products focused on this
  audience such as specialized computers (desktop or laptop, mouse,
  keyboard, monitor, etc.).
\item
  Electronidex also sells many of the products that were raised within a
  possible new portfolio for Blackwell (iMac is a clear example of
  this). In addition, these are among its best-selling products (as also
  was predicted in a previous report). We believe that this acquisition
  is aligned with the previous strategy set by Blackwell and that it
  would accelerate its implementation and minimize its risks.
\item
  Electronidex sold 2109 Acer products in the month analysed. Previously
  it was also raised whether to sell Acer or Sony products. The fact
  that they already sell one of these brands could mean some kind of
  strategic partnership for Blackwell.
\item
  For all the above, the acquisition of Electronidex by Blackwell is
  recommended.
\end{itemize}


\end{document}
